%%%%%%%%%%%%%%%%%%%%%%%%%%%%%%%%%%%%%%%%%%%%%%%%%%%%%%%%%%
% preamble.tex
% template-preamble
%
% Author: Carolin Zöbelein
% Email: contact@carolin-zoebelein.de
% PGP: D4A7 35E8 D47F 801F 2CF6 2BA7 927A FD3C DE47 E13B
%%%%%%%%%%%%%%%%%%%%%%%%%%%%%%%%%%%%%%%%%%%%%%%%%%%%%%%%%%
\documentclass{scrartcl}

\usepackage[headsepline,footsepline]{scrpage2}
\pagestyle{scrheadings}
\clearscrheadfoot
\setheadsepline{1.5pt}
\setfootsepline{1.5pt}

% Head
\ohead{\headmark}
\automark{section}

% Footer
\cfoot{\pagemark}

\usepackage[utf8x]{inputenc}
\usepackage[english]{babel}

\usepackage{footnote}
\usepackage{amssymb}
\usepackage{url}
\usepackage{graphicx}
\usepackage{amsmath}
\usepackage{hyperref}
%\usepackage{minitoc}
\usepackage{float}
\usepackage{longtable}
\usepackage{enumitem}

\usepackage{verbatim}	% For block comments


% Font style from resume
\usepackage{lmodern}
\renewcommand*\familydefault{\sfdefault}
\usepackage{mathptmx}%% Font Times


%\usepackage{fontawesome}
\usepackage{fontawesome5}


\usepackage{listings}
\lstset{
language=Python,
basicstyle=\small\sffamily,
%basicstyle=\tiny\sffamily,
numbers=left,
numberstyle=\tiny,
frame=tb,
%frame=single,
columns=fullflexible,
showstringspaces=false
}

\newtheorem{theorem}{Theorem}[section]
\newtheorem{lemma}[theorem]{Lemma}
\newtheorem{definition}[theorem]{Definition}
\newtheorem{example}[theorem]{Example}
\newtheorem{xca}[theorem]{Exercise}
\newtheorem{remark}[theorem]{Remark}

% Example: \authoremail{example@example.com, AAAA BBBB CCCC DDDD EEEE FFFF GGGG HHHH IIII JJJJ}
\newcommand{\authoremail}[2]{\textit{E-mail address:} \texttt{#1}, \textit{PGP Fingerprint}: \texttt{#2}}

% Example: \authorurl{http://www.example.com}
\newcommand{\authorurl}[1]{\textit{URL:} \url{#1}}

% Example: \keywords{keyword1, keyword2, keyword3}
\newcommand{\keywords}[1]{\textbf{Keywords:} #1}

% Example: \license{example license}
\newcommand{\license}[1]{\textbf{License:} #1}

% Example: \subjclass{2010}{Mathematics Subject Classification}{Primary 11N05}
\newcommand{\subjclass}[3]{\textbf{Subjclass:} #1 \textit{#2}. #3.}

% ======================================================================
\begin{document}
% ======================================================================
\begin{titlepage}
	%\begin{flushleft}
	%	\includegraphics[width=0.15\textwidth]{example-image-1x1}\par\vspace{1cm}
	%\end{flushleft}
	\begin{center}
	%{\scshape\LARGE Pony University \par}
	%\vspace{1cm}
    \begin{center}
            \includegraphics[width=0.2\textwidth]{images/logo_web.png}
    \end{center}
    \vspace{1cm}
	{\scshape\Large Research notes\par}
	\vspace{1.5cm}
	{\huge\bfseries Primes (part 01): Recursion basics\par}
	\vspace{2cm}
	{\Large\itshape Carolin Z\"obelein\footnote{PGP signing key (NOT for communication!): 8F31 C7C6 E67E 9ACE 8E12 E2EF 0DE3 A4D3 BA87 2A8B}\par}
	\vfill
	%supervised by\par
	%Dr.~John \textsc{Doe}
	\textsc{id: notes\_0001}\par
	\textsc{License: CC-BY-NC-ND}
	\vfill

	\fbox{\begin{minipage}{30em}
	Carolin Zöbelein\\
	Independent mathematical scientist

	\vspace{0.3cm}
	E-Mail: contact@carolin-zoebelein.de\\
	PGP: D4A7 35E8 D47F 801F 2CF6 2BA7 927A FD3C DE47 E13B\\
	Website: \url{https://research.carolin-zoebelein.de}
	\end{minipage}}

	\vfill

% Bottom of the page
	{\large Version: v01\par}
	{\large May 4, 2019\par}
	\end{center}
\end{titlepage}

% Suppress page numbers
%\thispagestyle{empty}
\pagenumbering{gobble}
% ======================================================================
% ----------------------------------------------------------------------
\newpage
\section*{Abstract}
\label{s:abstract}
% ----------------------------------------------------------------------
\begin{abstract}
	This notes give an overview over all necessary items for calculating prime numbers (only odds, we always ignore number $2$) recursively. We start, with the assumption that we know that $3$ and $5$ are prime numbers. We define equations which gave us all numbers which not belong to their time tables, do an intersection of this, and hence determine the next prime numbers from this. So we can do the same thing again. Taking all our knowing prime numbers, $3$, $5$ and the new ones, can define equations for the numbers which not belong to this time tables, do an intersection of this, and hence get the new prime numbers.\\
	This notes talks about all aspects which we have to consider to do so, and shows that we always have all necessary information to do this recursion without limitations. Since, we are also able to determine a closed analytical equation for the intersection of an arbitrary number of intersection, we are also able to do an recursive considering of it.\\
	In future work, it is still necessary to think about a nicer describing equation of this intersection, to find a better presentation of the final recursion formula.
\end{abstract}
% ----------------------------------------------------------------------
\section*{Content}
\label{s:content}
% ----------------------------------------------------------------------
\begin{enumerate}
	\item [I.] Page 1 - 4: Basic equations and solution of the non-times-table equations intersection
	\item [II.] Page 5 - 5: Relationship between old and new (modified) equation
	\item [III.] Page 6 - 7: Relationship between old and new (modified) equation - Floor corrections
	\item [IV.] Page 8 - 14: General form of the intersection of an arbitrary number of non-times-table equations
	\item [V.] Page 15 - 16: Non-times-table equation correction of $Y_{i}$ to the valid range, by substitution 
	\item [VI.] Page 17 - 20: Restriction on only odd solutions and the final, general intersection equation
	\item [VII.] Page 21 - 22: Verification that our new (modified) equation is also valid (can used without restrictions) during every recursion step
	\item [VIII.] Page 23 - 27: Determine valid $Y_{i}$ and $z_{i,j}$ ranges
	\item [IX.] Page 28 - 28: Approximations for Floor function
	\item [X.] Page 29 - 30: Let's put all together. ToDo.
\end{enumerate}
% ======================================================================
\end{document}
